% -*- root: ../root.tex -*-
\section{Fundamentels}

\subsection{Paving}

An arbitrary collection of subsets is a  \important{paving}

\subsection{Algebra}
\begin{defn}
A paving \(\mathbb{A}\)  on a set \(X\)  is called an  \important{algebra}
if

\begin{itemize}
  \item \(X\in\mathbb{A}\)
  \item \(A\in\mathbb{A}\Rightarrow A^{c}\in\A\)
  \item \(A,B\in\mathbb{A}\Rightarrow A\cup B\in\A\)
\end{itemize}
\end{defn}
\begin{lem}

If \(\mathbb{A}\)  is an \important{algebra} on \(X\) , then \(\emptyset\in\mathbb{A}\) \end{lem}

\begin{proof}
We know that \(X\)  itself is a member of \(\mathbb{A}\) , and we know that \(\mathbb{A}\)  is is stable under formation of complements. But the complements of \(X\)  is indeed \(\emptyset\) .
\end{proof}

\begin{lem}
If \(\A\)  is an \important{algebra} on \(X\) , is holds that
\[
A,B\in\A\Rightarrow A\cap B\in\A
\]
\end{lem}

\begin{proof}
Take \(A\)  and \(B\)  in \(\A\) . As \(\A\)  is stable under formation of complements, we see that \(A^{c}\)  and \(B^{c}\)  are two \(\A\) -sets. As \(\A\)  is stable under formation of unions, we set that \(A^{c}\cup B^{c}\in\A\) . If we take the complement again, we see that
\[
  A\cap B=(A^{c}\cup B^{c})^{c}\in\A
\]
using {de Morgan's law}
\end{proof}

\begin{lem}
If \(\A\)  is an algebra on \(X\) , it holds that
\[
  A,B\in\A\Rightarrow A\backslash B\in\A
\]
\end{lem}

\begin{proof}
Take \(A\)  and \(B\)  in \(\A\) . As \(\A\)  is stable under the formation of complements, we see that \(B^{c}\)  is in a \(\A\) -set. As \(\A\)  is stable under the formation of intersections, we see that \(A\cap B^{c}\in\A\) . Per definition of the set difference, we have that
\[
A\backslash B=A\cap B^{c}\in\A
\]
\end{proof}

\begin{lem}
If \(\A\)  is an algebra on \(\X\) , and \(\nset{A}\)  are sets in \(\A\) , is holds that
\[
\bigcup_{i=1}^{n}A_{i}\in\A,\:\bigcap_{i=1}^{n}A_{i}\in\A
\]
\end{lem}
\begin{proof}
For \(n=2\)  the claim is included in the definitation of an algebra. If the results is established for \(n-1\)  sets, we have
\[
\bigcup_{i=1}^{n}A_{i}=\left(\bigcup_{i=1}^{n-1}A_{i}\right)\cup A_{n}\in\A
\]
\end{proof}

\subsection{{sigma-algebras}}
\index{sigma algebra}
The concept of algebras does not work under under approximate schemes. Therefore we introduce \(\sigma\)-algrebras.

\begin{defn}
A paving \(\mathbb{E}\)  on a set \(\X\)  is called a \(\sigma\)-algebra if
\begin{itemize}
  \item \(\X\in\es\)
  \item \(A\in\es\Rightarrow A^{c}\in\es\)
  \item \(A_{1},A_{2},\dots\in\es\Rightarrow\bigcup_{i=1}^{\infty}A_{i}\in\es\)
\end{itemize}

\begin{defn}
Lad \(F\) være en abitrær familie a delmænger af \(\X\). Der eksistere en unique \emph{mindste} \(\sigal\) der indeholder alle mængder i \(F\) (\(F\) er ikke selv nødvendigvis en \(\sigal\)). Foreningenmængden af alle \(\sigal\) der indeholder F. Denne \(\sigal\) \(\sigma(F)\) er \(\sigal\)en genereret af F.
\end{defn}
A  \important{measurable space} is a pair \((\X,\es),\)  consisting
of the set \(\X\)  and a \(\sigma\) --algebra \(\es\)  on \(\X\) . We say that
a subset \(A\subset \X\)  is \(\es\) --measurable if \(A\in\es\)
\end{defn}
\begin{lem}
If \(\E\)  is an \(\sigma\)-algebra on \(\X\), the it is also an algebra.
\end{lem}
\begin{proof}
see book page 11.
\end{proof}
\subsection{Borel Sigma algebra} % (fold)
\label{sub:borel_sigma_algebra}
\begin{defn}
  The \important{Borel Sigma algebra}  \(\B\) is the smallest \(\sigal\) generated by the open sets. symboliccally \(\B=\sigma(\os)\)
\end{defn}
\begin{rem}
  As the borel algrebra \(\B\) is a \(\sigal\) which is stabel under the formation of complements, \(\B\) is also the \(sigal\) generated on the closed sets.
\end{rem}
\subsection{Important distributions} % (fold)
\label{sub:important_distrLJibutions}
\begin{table}
  \setlength\extrarowheight{10pt}
  \begin{tabular}{ >{$}l<{$}  >{$}l<{$} | >{$}l<{$}  >{$}l<{$}  }
  \toprule
    \text{Distributions}    & Continues                                                                                                                                                 & Discrete\tabularnewline
    \midrule
    \text{Uniform}          & \displaystyle\frac{1}{\beta}, \text{ for } x\in(\alpha,\alpha+\beta)                                                                                      & \text{Bionomial} & \displaystyle \begin{pmatrix} n \\ x\end{pmatrix}p^x(1-p)^{n-x} \tabularnewline
    \text{Exponential}      &\displaystyle \frac{1}{\beta}e^{-x/\beta}                                                                                                                  & \text{H-gemometric}& \displaystyle \begin{pmatrix}N_1 \\ x \end{pmatrix}\begin{pmatrix}N-N_1 \\ n-x \end{pmatrix}\Biggm/\begin{pmatrix} N \\N \end{pmatrix} \tabularnewline
    \text{Cauchy}           & \displaystyle\frac{1}{\pi(1+x^2)}                                                                                                                         & \text{Poisson} & \displaystyle\frac{\lambda^x}{x!}e^{-\lambda}\tabularnewline
    \text{Normal}           & \displaystyle\frac{1}{\sqrt{2\pi}}e^{-(x-\xi)^2/2\sigma^2}                                                                                                & &\tabularnewline
    \text{Gamma } \Gamma & \displaystyle\frac{1}{\beta^\lambda\Gamma(\lambda)}x^{1-\lambda}e^{-x/\beta}                                                                                 & &\tabularnewline
    \text{Beta}             & \displaystyle\frac{1}{\lambda_1,\lambda_2}x^{\lambda_1-1}(1-)^{\lambda_2-1}                                                                               & &\tabularnewline
    \text{F}                & \displaystyle\frac{\lambda_1^{\lambda_1}\lambda_2^{\lambda_2}}{B(\lambda_1,\lambda_2)}\frac{x^{\lambda_1-1}}{(\lambda_1x+\lambda_2)^{\lambda_+\lambda_2}} & &\tabularnewline
    \text{T}                & \displaystyle\frac{1}{\sqrt{2\lambda}B(\lambda,\frac{1}{2})}\frac{1}{\left(1+\frac{x^2}{2\lambda}\right)^{\lambda+1/2}}                                   & &\tabularnewline

    \bottomrule
  \end{tabular}
  \caption{distributions}
  \label{tbl:label}
\end{table}

% subsection important_distributions (end)
% subsubsection borel_sigma_algebra (end)