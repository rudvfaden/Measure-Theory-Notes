% -*- root: ../root.tex -*-
\section[Lecture 4]{Image measures} % (fold)
\label{sec:image_measures}
\begin{defn}[Image measure]
For two measurable spavces \((\X,\E)\)  with measure \(\mu\) and \(\Y,\K\). The \important{image measure} \(t(\mu)\), as the measure on \(\Y,\K\) is given by
\begin{align}
    t(\mu)=\mu\left(t^{-1}(B)\right)
\end{align}
\end{defn}
\subsection{Intergration with respect ot image measures} % (fold)
\label{sub:intergration_with_respect_ot_image_measures}
\begin{defn}[\index{Abstract-change-of-variable formula}Abstract-change-of-variable formula]
Let \(\X,\E,\mu\) be a measure space and let \((\Y,\K)\)  be a measurable space, and let \(t: \X\longrightarrow\Y\) be \(\E-\K\) measurable. For every function \(g\in\M^+(\Y,\K)\) it holds that
\begin{equation}
      \int g \dif t(\mu)=\int g\circ t \dif \mu
\end{equation}
\end{defn}
\begin{rem}
The above also holds for real-valued functions, but we have to make sure that the \(g\) function is \(t(\mu)\)-integrable. This holds iff \(g\circ t\)  is \(t(\mu)\)-integrable. See Collary 10.9 in EH.
\end{rem}
% subsection intergration_with_respect_ot_image_measures (end)
\subsection{Translation invariance} % (fold)
\label{sub:translation_invariance}
\begin{defn}
	A \important{translation} in \(\R^k\) is a map \(\R^k \longrightarrow \R^k\) of the form
	\begin{align}
	    \tau(x)=x+w\, \forall \, x\in\R^k
	\end{align}
	for fixed \(w\in\R^k\)
\end{defn}

A map is \important{translation invariant} if
\begin{align}
    \tau_w(\mu)=\mu \text{ for every choice of } w\in\R^k
\end{align}

\begin{rem}
Intuativly it does translation invariance means that the measure is determined by the shape of the set and not where it is located in \(\R^k\)
\end{rem}
\begin{rem}
The lebesque measure is translation invariant. See EH theorem 10.11
\end{rem}
% subsection translation_invariance (end)

% section image_measures (end)